%% LyX 2.0.4 created this file.  For more info, see http://www.lyx.org/.
%% Do not edit unless you really know what you are doing.
\documentclass[a4paper,english]{scrartcl}
\usepackage[T1]{fontenc}
\usepackage[latin9]{inputenc}
\usepackage{listings}
\lstset{aboveskip=2mm,
backgroundcolor={\color{lightgray}},
basicstyle={\small\ttfamily},
belowskip=2mm,
breakatwhitespace=true,
breaklines=true,
captionpos=b,
columns=flexible,
commentstyle={\color{darkgreen}},
frame=tb,
frame=single,
keywordstyle={\textbf},
keywordstyle={[2]\color{red}},
numbers=none,
numberstyle={\tiny\color{mauve}},
showstringspaces=false,
stringstyle={\color{dkgreen}},
tabsize=2}
\setcounter{secnumdepth}{2}
\usepackage[pdftex]{color}

\makeatletter

%%%%%%%%%%%%%%%%%%%%%%%%%%%%%% LyX specific LaTeX commands.
\pdfpageheight\paperheight
\pdfpagewidth\paperwidth


\@ifundefined{date}{}{\date{}}
%%%%%%%%%%%%%%%%%%%%%%%%%%%%%% User specified LaTeX commands.
\usepackage{color}
\definecolor{gray}{rgb}{0.5,0.5,0.5}
\definecolor{lightgray}{rgb}{0.95,0.95,0.95}
\definecolor{darkishgray}{rgb}{0.4,0.4,0.4}
\definecolor{darkgray}{rgb}{0.35,0.35,0.35}
\definecolor{darkgreen}{rgb}{0.25,0.45,0.25}
\definecolor{dkgreen}{rgb}{0,0.3,0}
\definecolor{mauve}{rgb}{0.38,0,0.62}
\definecolor{red}{rgb}{0.58,0,0.08}
\lstdefinelanguage{Scala}{
morekeywords={},
morekeywords=[2]{
public, private, protected,
abstract,case,catch,class,def,
do,else,extends,false,final,finally,
for,if,implicit,import,match,mixin,
new,null,object,override,package,
private,protected,requires,return,sealed,
super,this,throw,trait,true,try,
type,val,var,while,with,yield,
lazy,evt,observable,imperative},
otherkeywords={=>,<-,<\%,<:,>:,\#,@},
sensitive=true,
morecomment=[l]{//},
morecomment=[n]{/*}{*/},
morestring=[b]",
morestring=[b]',
morestring=[b]""",
basicstyle=\ttfamily\footnotesize 
}

\makeatother

\usepackage{babel}
\begin{document}

\title{\noindent REScala: ``Animal'' case study observations}


\author{\noindent Gerold Hintz}

\maketitle

\section{Advantages of Signals \& Events\protect \\
(vs events-only / signal-only)}


\subsection{Main point: detection of changes}

We need the combination of events and signals to model processes which
depend on the change of a value.\textbf{ }

\textbf{Example:} The germination of a plant is defined through a
process of aging, growing, and reaching a maximum size. This can be
expressed very concise with signals:

\begin{lstlisting}[language=Scala]
class Plant extends BoardElement {  
  
  val age = world.time.hour.changed.iterate(0)(_ + 1)
  val grows = age.changed && { _ % Plant.GrowTime == 0}
  val size = grows.iterate(0)(acc => math.min(Plant.MaxSize, acc + 1))
  val expands = size.changedTo(Plant.MaxSize)  
  
  expands += {_ =>    
    germinate() // spawn a new plant in proximity to this one
  }
}
\end{lstlisting}


The equivalent event-based code has to do a manual check on every
update:

\begin{lstlisting}[language=Scala]
class Plant extends BoardElement {

  var age = 0
  var size = 0  
  val grows = new ImperativeEvent[Unit]
  val expands = grows && (_ => size == Plant.MaxSize)

  expands += {_ => 
    germinate() // spawn a new plant in proximity to this one
 }

 tickHandler = {_: Unit => 
    age += 1
    if(age % Plant.GrowTime == 0){
      val oldSize = size
      size = math.min(Plant.MaxSize, size + 1)
      if(size != oldSize)
        grows()
    }
  }
}
\end{lstlisting}


The equivalent signal-only code would have to perform a manual check
for a value change as well. In the absence of events (in particular
the \texttt{changed} event), the relationship would have to be defined
as a pure functional dependancy, rather than through an explicit \texttt{grow}
event. The code in the \texttt{tick} function has to do a lot of ugly
manual checks. In addition, the reaction to a very specific condition
(size has reached a maximum value), gets cluttered into the \texttt{tick}
method, which should not have anything to do with that.

\begin{lstlisting}[language=Scala]
class Plant(override implicit val world: World) extends BoardElement {
  
  val age = Var(0)
  val size = Signal { math.min(Plant.MaxSize, age() / Plant.GrowTime) }    

  def tick {
     // we have to store the old size now, otherwise we could not detect changes 
    val oldSize = size.getValue
    
    age() = age.getValue + 1    
       
    if(size.getValue != oldSize){ // did the value change
       if(size.getValue == Plant.MaxSize) // did the value reach MaxSize
	      germinate() // spawn a new plant in proximity to this one
    }
  }
}
\end{lstlisting}



\section{Shortcomings of Signal code}


\subsection{Handlers on late bound events}

Sometimes we want to define an event on a signal which is late bound.
This works (but we have to make the event lazy). However, we can not
\emph{register an event handler} on this event. When the object \texttt{Animal}
gets instanciated, the signal \texttt{isDead} is still unbound.

\begin{lstlisting}[language=Scala]
abstract class Animal {
  
  val isDead: Signal[Boolean] // this value is abstract
  
  lazy val dies = isDead changedTo true  // we can do this
  dies += {_ => world.board.clear(position.getValue)} // we can not do this
}

class Carnivore extends Animal {
  isDead = Signal { energy() > 10} // subclass substitutes concrete signal
}
\end{lstlisting}



\subsection{Overriding signal values}

We can override any signal value in a subclass. However, as a signal
is a \texttt{val} member, we can not access the super-value. Consider
the class Animal with member energyDrain. In the subtype Female, we
want to override this value, by multipliying this signal with a given
factor. However

\begin{lstlisting}[language=Scala]
abstract class Animal {
    val energyDrain = Signal {...}
}

trait Female extends Animal {
	val isPregnant = Signal {...}
	val factor = Signal { if(isPregant()) 1.2 else 1 } // multiply energy drain by 1.2 if pregnant.	
	override val energyDrain = Signal { super.energyDrain() * factor() } // problem: super can not be used on val members!
 }
\end{lstlisting}

\end{document}
